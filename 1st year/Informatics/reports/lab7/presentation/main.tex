documentclass{beamer}

usepackage[utf8]{inputenc}
usepackage[russian]{babel}
usepackage{xcolor,colortbl}
usepackage[black, white, blue, red, magenta]{color}
newcommand{specialcell}[2][c]{% 
begin{tabular}[#1]{@{}c@{}}#2end{tabular}}

title{Презентация.Лабраторная работа №7 }
author{Сенина Мария P3112}
institute{ITMO University}
date{2020}
logo{includegraphics[height=1.5cm]{itmo.jpg}}

begin{document}

frame{titlepage}

begin{frame}
frametitle{Проблемы округления чисел}
Классическое правило округления - к ближайшему целому.
medbreak

begin{minipage}{.49textwidth}
  centering
    begin{tabular}{ccc}
    hline
    & Число & Округл.  hline
     & 1,1 & 1,0  hline
     & 2,9 & 3,0  hline
     & 5,0 & 5,0  hline
     & 3,4 & 3,0  hline
     & 8,6 & 9,0  hline
    Сумма & $textbf{21,0}$ & $textbf{21,0}$  hline
    end{tabular}
end{minipage}
begin{minipage}{.49textwidth}
  centering
      begin{tabular}{ccc}
    hline
    & Число & Округл.  hline
     & 1,5 & 2,0  hline
     & 2,5 & 3,0  hline
     & 5,5 & 6,0  hline
     & 3,5 & 4,0  hline
     & 8,5 & 9,0  hline
    Сумма & $textbf{21,5}$ & $textbf{24,0}$  hline
    end{tabular}
end{minipage}
end{frame}

begin{frame}{Пример накопленной ошибки округления}
begin{minipage}{.49textwidth}
  centering
  definecolor{Black}{rgb}{1,1,1}
  begin{tabular}{cc}
        hline
    cellcolor{black} textcolor{white}{Число} & cellcolor{black} textcolor{white}{Округл.}  hline
    rowcolor{Black}
    0,33 & 0  hline
    2,5 & 1  hline
    5,5 & 1  hline
    3,5 & 0  hline
    ... & ...  hline
    textcolor{blue}{0,50} & textcolor{blue}{1}  hline
    0,73 & 1  hline
    0.20 & 0  hline
    $textbf{Сумма1}$ & $textbf{Сумма2}$  hline
end{tabular}
end{minipage} 
begin{minipage}{.49textwidth}
  centering
В первом столбце из 100 возможных значений только одно приводит к накоплению ошибки в 50 коп., поэтому в среднем (Сумма2 – Сумма1)
 = $(frac{10000}{100}) cdot 50$коп. = 50 руб. переплаты!
 end{minipage}
end{frame}

begin{frame}{Проблемы округления чисел в различных СС}
    В системах счисления с textcolor{blue}{чётным} основанием накапливается ошибка округления
    begin{tabular}{c c c}
        underline{Основание 10} & $textbf{1, 2, 3, 4,}$ & ← округление в меньшую сторону 
         & $textbf{5, 6, 7, 8, 9,}$ &  ← округление в бóльшую сторону 
         & $textbf{0}$ & ← нет ошибки округления
    end{tabular}
    
medbreak
В системах счисления с textcolor{blue}{нечётным} основанием этой проблемы нет
begin{tabular}{c c c}
        underline{Основание 7} & $textbf{1, 2, 3,}$ & ← округление в меньшую сторону 
         & $textbf{4, 5, 6,}$ &  ← округление в бóльшую сторону 
         & $textbf{0}$ & ← нет ошибки округления
    end{tabular}
    
medbreak

textit{Актуальна ли проблема накопления ошибки округления для симметричных СС}
end{frame}

begin{frame}{Решение проблемы с округлением
в СС с чётным основанием}
small
Суть решения — использовать неклассические правила округления
begin{itemize}
    item textcolor{blue}{Случайное округление} используется датчик случайных чисел при принятии решения о том, в бóльшую или меньшую сторону следует округлять. 
    item textcolor{blue}{Банковское округление} (к ближайшему чётному) $3,5 approx 4$, но $2,5 approx 2$.
    item textcolor{blue}{К ближайшему нечётному} $3,5 approx 3$, но $2,5 approx 3$. Аналогично $4,3_{(6)} approx 5_{(6)}$.
    item textcolor{blue}{Чередующееся} направление округления меняется на противоположное при
каждой операции округления (необходимо «помнить» о предыдущем округлении).
end{itemize}

medbreak
$textbf{Примечание.}$ Каждое из правил можно применять как полностью универсально, так и комбинировано с классическим правилом округления, дополняя его лишь при округлении пограничных значений.
end{frame}

begin{frame}{Пример округления}

begin{tabular}{p{1.5cm}p{1.15cm}{1.15cm}p{1.15cm}p{1.15cm}p{1.15cm}}
hline
rowcolor[Black]{.9}textcolor{white}{Number, NS(2)} & textcolor{white}{Math} & textcolor{white}{To odd} & textcolor{white}{To even} & textcolor{white}{Random Cion test} & textcolor{white}{Striped}  hline
rowcolor[gray]{.9} 10,1 & 11 & 11 & 10 & 11 & 10  hline
00,1 & 01 & 01 & 00 & 01 & 01  hline
rowcolor[gray]{.9}01,1 & 01 & 01 & 01 & 01 & 01  hline
10,0 & 10 & 10 & 10 & 10 & 10  hline
rowcolor[gray]{.9}11,0 & 11 & 11 & 11 & 11 & 11  hline
00,1 & 01 & 01 & 00 & 00 & 00  hline
rowcolor[gray]{.9}01,1 & 10 & 01 & 10 & 01 & 10  hline
00,0 & 00 & 00 & 00 & 00 & 00  hline
rowcolor[Black]{.9}textcolor{white}{Sum} & & & & &  hline
rowcolor[gray]{.9}1011 & 1101 & 1100 & 1010 & 1011 & 1011  hline
end{tabular}
    
end{frame}


end{document}